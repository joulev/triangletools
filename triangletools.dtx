% \iffalse
%<*driver>
\documentclass[11pt]{l3doc}
\ExplSyntaxOn
\cs_set_protected:Npn \__codedoc_function_typeset:
  {
    \dim_zero:N \l__codedoc_trial_width_dim
    \hcoffin_set:Nn \l__codedoc_functions_coffin { \__codedoc_typeset_functions: }
    \dim_set:Nn \l__codedoc_trial_width_dim
      { \box_wd:N \l__codedoc_functions_coffin }
    \bool_set_false:N \l__codedoc_long_name_bool
  }
\ExplSyntaxOff
\renewcommand\MacroLongFont{\ttfamily\small}
\let\oldsyntax\syntax
\let\oldendsyntax\endsyntax
\renewenvironment{syntax}{^^A
  \oldsyntax^^A
  \renewcommand\meta[1]{$\langle${\itshape\rmfamily##1}$\rangle$}^^A
  \renewcommand\marg[1]{\texttt{\{}\meta{##1}\texttt{\}}}^^A
  \renewcommand\oarg[1]{\texttt{[}\meta{##1}\texttt{]}}^^A
  \renewcommand\parg[1]{\texttt{(}\meta{##1}\texttt{)}}^^A
}{\oldendsyntax}
\usepackage{xparse}
\usepackage[margin=1cm,left=7cm,bottom=1.5cm]{geometry}
\usepackage[english]{babel}
\usepackage{libertine}
\usepackage[libertine]{newtxmath}
\usepackage[scaled=0.75]{beramono}
\usepackage[T1]{fontenc}
\usepackage{triangletools}
\usepackage[listings]{tcolorbox}
\definecolor{maincolor}{HTML}{6830C9}
\hypersetup{linkcolor=maincolor}
\newtcblisting{codeexample}[1][]{size=minimal,
  sidebyside gap=1cm,lefthand width=6.5cm,lower separated=false,
  fontupper=\raggedleft,colback=white,colframe=white,
  verbatim ignore percent,grow to left by=7cm,text side listing,
  listing options={basicstyle=\ttfamily\small,keywordstyle=\ttfamily\small,
  numbers=left,numberstyle=\sffamily\tiny\color{gray},stepnumber=1,numbersep=5pt},#1}
\pgfkeys{/tcb/.cd,code only/.code={\pgfkeysalso{listing only,grow to left by=-5mm}}}
\newcommand\tikzname{Ti\emph{k}Z}
\setlength\parindent{0pt}
\EnableCrossrefs
\CodelineIndex
\RecordChanges
\begin{document}
\DocInput{triangletools.dtx}
\end{document}
%</driver>
% \fi
%
%
%
% \title{The \pkg{triangletools} package}
% \author{Vu Van Dung}
% \date{\today}
% \maketitle
%
% \tableofcontents
% \setlength\parskip{1ex}^^A You have to wait until the TOC is done
%
% \begin{documentation}
%
%\section{Introduction}
%This package aims to help you construct special points in a triangle directly 
%in a short and easy way. Using this package, you can construct most important 
%points listed in Clark Kimberling's Encyclopedia of Triangle Centers (ETC). 
%Currently, all points numbered from $X_1$ and $X_{10}$, as well as the excenter, 
%are supported; however with other utilities in this package (see Section 
%\ref{utilities}) and a bit of knowledge in geometry and \pkg{expl3} programming, 
%you can construct even more.
%\section{Loading the package}
%This package can be loaded as usual.
%\begin{codeexample}[code only]
%\usepackage{triangletools}
%\end{codeexample}
%It will load \tikzname\ and \pkg{expl3} automatically.
%\section{User interface}
%The user interface of this package, including that of the utilities, is 
%provided as \pkg{pgf} keys under the tree |/tikz/triangletools|.
%
%Note that, in the following sections, a \emph{coordinate} means a \emph{named} 
%\tikzname\ coordinate. That is, in the following example,
%\begin{codeexample}[code only]
%\begin{tikzpicture}
%  \draw (0,0) -- (3,0) coordinate (a);
%\end{tikzpicture}
%\end{codeexample}
%|a| is a named coordinate, while |0,0| or |3,0| are \emph{not} named coordinates.
%The current implementation of this package only allows named coordinates in the 
%user interface. It is like the |angles| \tikzname\ library.
%\subsection{Accessing the keys}
%\begin{function}{trt}
%  \begin{syntax}
%    /tikz/trt=\marg{keys}
%  \end{syntax}
%  It executes \emph{keys} with the key path set to |/tikz/triangletools|, which 
%  is the main key tree of this package.
%
%  This key is used to access all other keys in the user interface.
%\end{function}
%\subsection{Circles associated with triangle centers}
%\begin{function}{\trtradius}
%  Some points, for example the incenter and the circumcenter, are associated 
%  with some special circles. If the requested point is associated with a circle, 
%  this macro stores the radius of that circle, in points (|pt|).
%
%  This macro is assigned \emph{globally} every time a point is requested. 
%  Therefore, it stores the radius related to the last point that has a circle.
%  So beware that while it always gives you some values once you have drawn 
%  such points, that value might not be what you want. It is recommended to 
%  use this macro \emph{immediately after} the execution of triangle center 
%  keys.
%
%  In Section \ref{triangle-centers}, if a point has a \cs{trtradius} 
%  associated to it, the circle will be drawn in the code example. Currently 
%  $X_1$, the excenter, $X_3$, $X_5$ and $X_{10}$ can change the value of 
%  \cs{trtradius}.
%
%  If the macro is used before any center with a circle is constructed, an 
%  error message will be issued.
%\end{function}
%\subsection{Triangle centers}\label{triangle-centers}
%\begin{function}{incenter, \trt_sp_incenter:nnnn}
%  \begin{syntax}
%    /tikz/triangletools/incenter=\parg{coor 1}\parg{coor 2}\parg{coor 3}
%    \cs{trt_sp_incenter:nnnn} \marg{coor 1}\marg{coor 2}\marg{coor 3}\marg{name}
%  \end{syntax}
%  Find the incenter $X_1$ of the triangle joining \tikzname\ coordinates 
%  \meta{coor 1}, \meta{coor 2} and \meta{coor 3}. The incenter is saved to 
%  \tikzname\ coordinate \meta{name}.
%
%  If you use the key (why do you use the function anyway), \meta{name} is set 
%  to |trt output| by default. You can change that using |output name|, see 
%  Section \ref{customization}.
%\end{function}
%\begin{codeexample}[]
%\begin{tikzpicture}
%  \draw (1,3) coordinate (a) -- 
%        (0,0) coordinate (b) -- 
%        (4,0) coordinate (c) -- cycle;
%  \fill[maincolor,trt={incenter=(a)(b)(c)}] (trt output) 
%    circle[radius=1.5pt] node[below] {$I$};
%  \draw[maincolor] (trt output) circle (\trtradius);
%\end{tikzpicture}
%\end{codeexample}
%\begin{function}{excenter, \trt_sp_excenter:nnnn}
%  \begin{syntax}
%    /tikz/triangletools/excenter=\parg{coor 1}\parg{coor 2}\parg{coor 3}
%    \cs{trt_sp_excenter:nnnn} \marg{coor 1}\marg{coor 2}\marg{coor 3}\marg{name}
%  \end{syntax}
%  Find the excenter of the triangle. The returned point will be on the 
%  internal angular bisector at \meta{coor 1}. Note that the order matters: 
%  |excenter=(a)(b)(c)| is \emph{different} from |excenter=(b)(a)(c)|.
%\end{function}
%\begin{codeexample}[]
%\begin{tikzpicture}
%  \draw (.5,3) coordinate (a) -- 
%        (0 ,0) coordinate (b) -- 
%        (2 ,0) coordinate (c) -- cycle;
%  \fill[maincolor,trt={excenter=(a)(b)(c)}] (trt output) 
%    circle[radius=1.5pt] node[below] {$I_a$};
%  \draw[maincolor] (trt output) circle (\trtradius);
%\end{tikzpicture}
%\end{codeexample}
%\begin{function}{centroid, \trt_sp_centroid:nnnn}
%  \begin{syntax}
%    /tikz/triangletools/centroid=\parg{coor 1}\parg{coor 2}\parg{coor 3}
%    \cs{trt_sp_centroid:nnnn} \marg{coor 1}\marg{coor 2}\marg{coor 3}\marg{name}
%  \end{syntax}
%  Find the centroid $X_2$ of the triangle.
%\end{function}
%\begin{codeexample}[]
%\begin{tikzpicture}
%  \draw (1,3) coordinate (a) -- 
%        (0,0) coordinate (b) -- 
%        (4,0) coordinate (c) -- cycle;
%  \fill[maincolor,trt={centroid=(a)(b)(c)}] (trt output) 
%    circle[radius=1.5pt] node[below] {$G$};
%\end{tikzpicture}
%\end{codeexample}
%\begin{function}{circumcenter, \trt_sp_circumcenter:nnnn}
%  \begin{syntax}
%    /tikz/triangletools/circumcenter=\parg{coor 1}\parg{coor 2}\parg{coor 3}
%    \cs{trt_sp_circumcenter:nnnn} \marg{coor 1}\marg{coor 2}\marg{coor 3}\marg{name}
%  \end{syntax}
%  Find the circumcenter $X_3$ of the triangle.
%\end{function}
%\begin{codeexample}[]
%\begin{tikzpicture}
%  \draw (1,3) coordinate (a) -- 
%        (0,0) coordinate (b) -- 
%        (4,0) coordinate (c) -- cycle;
%  \fill[maincolor,trt={circumcenter=(a)(b)(c)}] (trt output) 
%    circle[radius=1.5pt] node[below] {$O$};
%  \draw[maincolor] (trt output) circle (\trtradius);
%\end{tikzpicture}
%\end{codeexample}
%\begin{function}{orthocenter, \trt_sp_orthocenter:nnnn}
%  \begin{syntax}
%    /tikz/triangletools/orthocenter=\parg{coor 1}\parg{coor 2}\parg{coor 3}
%    \cs{trt_sp_orthocenter:nnnn} \marg{coor 1}\marg{coor 2}\marg{coor 3}\marg{name}
%  \end{syntax}
%  Find the orthocenter $X_4$ of the triangle.
%\end{function}
%\begin{codeexample}[]
%\begin{tikzpicture}
%  \draw (1,3) coordinate (a) -- 
%        (0,0) coordinate (b) -- 
%        (4,0) coordinate (c) -- cycle;
%  \fill[maincolor,trt={orthocenter=(a)(b)(c)}] (trt output) 
%    circle[radius=1.5pt] node[below] {$H$};
%\end{tikzpicture}
%\end{codeexample}
%\begin{function}{triangle center}
%  \begin{syntax}
%    /tikz/triangletools/triangle center=\parg{coor 1}\parg{coor 2}\parg{coor 3}\parg{index}
%  \end{syntax}
%  Find the point $X_{\text{\meta{index}}}$ of the triangle. Currently \meta{index}
%  can be any integer between and including $1$ and $10$.
%\end{function}
%\begin{function}{\trt_sp_ninepointcenter:nnnn}
%  \begin{syntax}
%    /tikz/triangletools/triangle center=\parg{coor 1}\parg{coor 2}\parg{coor 3}(5)
%    \cs{trt_sp_ninepointcenter:nnnn} \marg{coor 1}\marg{coor 2}\marg{coor 3}\marg{name}
%  \end{syntax}
%  Find the nine-point center $X_5$ of the triangle.
%\end{function}
%\begin{codeexample}[]
%\begin{tikzpicture}
%  \draw (1,3) coordinate (a) -- 
%        (0,0) coordinate (b) -- 
%        (4,0) coordinate (c) -- cycle;
%  \fill[maincolor,trt={triangle center=(a)(b)(c)(5)}] (trt output) 
%    circle[radius=1.5pt] node[below] {$X_5$};
%  \draw[maincolor] (trt output) circle (\trtradius);
%\end{tikzpicture}
%\end{codeexample}
%\begin{function}{\trt_sp_symmedian:nnnn}
%  \begin{syntax}
%    /tikz/triangletools/triangle center=\parg{coor 1}\parg{coor 2}\parg{coor 3}(6)
%    \cs{trt_sp_symmedian:nnnn} \marg{coor 1}\marg{coor 2}\marg{coor 3}\marg{name}
%  \end{syntax}
%  Find the symmedian point $X_6$ (\emph{aka.} the Lemoine point or Grebe point) 
%  of the triangle.
%\end{function}
%\begin{codeexample}[]
%\begin{tikzpicture}
%  \draw (1,3) coordinate (a) -- 
%        (0,0) coordinate (b) -- 
%        (4,0) coordinate (c) -- cycle;
%  \fill[maincolor,trt={triangle center=(a)(b)(c)(6)}] (trt output) 
%    circle[radius=1.5pt] node[below] {$X_6$};
%\end{tikzpicture}
%\end{codeexample}
%\begin{function}{\trt_sp_gergonne:nnnn}
%  \begin{syntax}
%    /tikz/triangletools/triangle center=\parg{coor 1}\parg{coor 2}\parg{coor 3}(7)
%    \cs{trt_sp_gergonne:nnnn} \marg{coor 1}\marg{coor 2}\marg{coor 3}\marg{name}
%  \end{syntax}
%  Find the Gergonne point $X_7$ of the triangle.
%\end{function}
%\begin{codeexample}[]
%\begin{tikzpicture}
%  \draw (1,3) coordinate (a) -- 
%        (0,0) coordinate (b) -- 
%        (4,0) coordinate (c) -- cycle;
%  \fill[maincolor,trt={triangle center=(a)(b)(c)(7)}] (trt output) 
%    circle[radius=1.5pt] node[below] {$X_7$};
%\end{tikzpicture}
%\end{codeexample}
%\begin{function}{\trt_sp_nagel:nnnn}
%  \begin{syntax}
%    /tikz/triangletools/triangle center=\parg{coor 1}\parg{coor 2}\parg{coor 3}(8)
%    \cs{trt_sp_nagel:nnnn} \marg{coor 1}\marg{coor 2}\marg{coor 3}\marg{name}
%  \end{syntax}
%  Find the Nagel point $X_8$ of the triangle.
%\end{function}
%\begin{codeexample}[]
%\begin{tikzpicture}
%  \draw (1,3) coordinate (a) -- 
%        (0,0) coordinate (b) -- 
%        (4,0) coordinate (c) -- cycle;
%  \fill[maincolor,trt={triangle center=(a)(b)(c)(8)}] (trt output) 
%    circle[radius=1.5pt] node[below] {$X_8$};
%\end{tikzpicture}
%\end{codeexample}
%\begin{function}{\trt_sp_mittenpunkt:nnnn}
%  \begin{syntax}
%    /tikz/triangletools/triangle center=\parg{coor 1}\parg{coor 2}\parg{coor 3}(9)
%    \cs{trt_sp_mittenpunkt:nnnn} \marg{coor 1}\marg{coor 2}\marg{coor 3}\marg{name}
%  \end{syntax}
%  Find the \emph{mittenpunkt} $X_9$ of the triangle.
%\end{function}
%\begin{codeexample}[]
%\begin{tikzpicture}
%  \draw (1,3) coordinate (a) -- 
%        (0,0) coordinate (b) -- 
%        (4,0) coordinate (c) -- cycle;
%  \fill[maincolor,trt={triangle center=(a)(b)(c)(9)}] (trt output) 
%    circle[radius=1.5pt] node[below] {$X_9$};
%\end{tikzpicture}
%\end{codeexample}
%\begin{function}{\trt_sp_spieker:nnnn}
%  \begin{syntax}
%    /tikz/triangletools/triangle center=\parg{coor 1}\parg{coor 2}\parg{coor 3}(10)
%    \cs{trt_sp_spieker:nnnn} \marg{coor 1}\marg{coor 2}\marg{coor 3}\marg{name}
%  \end{syntax}
%  Find the Spieker center $X_{10}$ of the triangle.
%\end{function}
%\begin{codeexample}[]
%\begin{tikzpicture}
%  \draw (1,3) coordinate (a) -- 
%        (0,0) coordinate (b) -- 
%        (4,0) coordinate (c) -- cycle;
%  \fill[maincolor,trt={triangle center=(a)(b)(c)(10)}] (trt output) 
%    circle[radius=1.5pt] node[below] {$X_{10}$};
%  \draw[maincolor] (trt output) circle (\trtradius);
%\end{tikzpicture}
%\end{codeexample}
%\subsection{Other utilities}\label{utilities}
%\subsubsection{Line tools}\label{utilities.linetools}
%The line tools utility can helps you play with some (very basic) operations 
%related to lines.
%\begin{function}{intersection}
%  \begin{syntax}
%    /tikz/triangletools/intersection=\parg{coor 1}\parg{coor 2}|--|\parg{coor 3}\parg{coor 4}
%  \end{syntax}
%  There are two lines, the first joins \meta{coor 1} and \meta{coor 2}, and the 
%  other joins \meta{coor 3} and \meta{coor 4}. This finds the intersection of 
%  these lines.
%
%  If the two lines are parallel, |trt output| is set to |(0,0)|, and the 
%  package will report a warning.
%\end{function}
%\begin{codeexample}[]
%\begin{tikzpicture}
%  \draw (0,0) coordinate (a) -- (0,1) coordinate (b);
%  \draw (1,0) coordinate (c) -- (.5,1) coordinate (d);
%  \coordinate (e) at (1,1);
%  \fill[maincolor,trt={intersection=(a)(b)--(c)(d)}] (trt output) 
%    circle (1.5pt) node[above] {$I$};
%  \fill[maincolor,trt={intersection=(a)(b)--(c)(e)}] (trt output)
%    circle (1.5pt) node[left] {$J$};
%\end{tikzpicture}
%\end{codeexample}
%\begin{function}{foot of perpendicular}
%  \begin{syntax}
%    /tikz/triangletools/foot of perpendicular=\parg{coor 1}|--|\parg{coor 2}\parg{coor 3}
%  \end{syntax}
%  Find the foot of perpendicular of point \meta{coor 1} to the line joining 
%  \meta{coor 2} and \meta{coor 3}.
%\end{function}
%\begin{codeexample}[]
%\begin{tikzpicture}
%  \draw (0,0) coordinate (b) -- (4,1) coordinate (c);
%  \fill (1,2) circle (1.5pt) coordinate (a);
%  \fill[maincolor,trt={foot of perpendicular=(a)--(b)(c)}] (trt output)
%    circle (1.5pt) node[below] {$H$};
%\end{tikzpicture}
%\end{codeexample}
%You can do much more using these \pkg{expl3} functions.
%\begin{function}{\trt_lt_get_line_equation:nnNNN}
%  \begin{syntax}
%    \cs{trt_lt_get_line_equation:nnNNN}\marg{coor 1}\marg{coor 2}\meta{a}\meta{b}\meta{c}
%  \end{syntax}
%  Find the equation of the line joining \meta{coor 1} and \meta{coor 2}, in the 
%  form of $ax+by=c$. The \pkg{l3fp} \emph{local} variables \meta{a}, \meta{b} 
%  and \meta{c} will be set accordingly.
%
%  Note that for any pair of points \meta{coor 1} and \meta{coor 2}, there are 
%  infinitely many solutions for \meta{a}, \meta{b} and \meta{c}. This function 
%  will produce one of such solution. While the solution is likely to be the 
%  simplest of all possible ones, this is not guaranteed.
%\end{function}
%\begin{codeexample}[]
%\begin{tikzpicture}
%  \coordinate (a) at (1,0);
%  \coordinate (b) at (4,1);
%  \ExplSyntaxOn
%  \fp_new:N \l_foo_tmpa_fp
%  \fp_new:N \l_foo_tmpb_fp
%  \fp_new:N \l_foo_tmpc_fp
%  \trt_lt_get_line_equation:nnNNN {a} {b} 
%    \l_foo_tmpa_fp \l_foo_tmpb_fp \l_foo_tmpc_fp
%  \def \resultequation {
%    $\fp_eval:n {round (\l_foo_tmpa_fp / 1cm)}x + 
%     \fp_eval:n {round (\l_foo_tmpb_fp / 1cm)}y
%    =\fp_eval:n {round (\l_foo_tmpc_fp / (1cm * 1cm))}$
%  }
%  \ExplSyntaxOff
%  \draw (a) -- (b) node[midway,sloped,below] {\resultequation};
%  \ExplSyntaxOn
%  \path (1,-1) node[below,align=left] {
%    Actual ~ values:\\
%    $a = \fp_eval:n {\l_foo_tmpa_fp / 1cm}$\\
%    $b = \fp_eval:n {\l_foo_tmpb_fp / 1cm}$\\
%    $c = \fp_eval:n {\l_foo_tmpc_fp / (1cm * 1cm)}$
%  };
%  \ExplSyntaxOff
%\end{tikzpicture}
%\end{codeexample}
%\ExplSyntaxOn
%\cs_undefine:N \l_foo_tmpa_fp
%\cs_undefine:N \l_foo_tmpb_fp
%\cs_undefine:N \l_foo_tmpc_fp
%\ExplSyntaxOff
%\begin{function}{\trt_lt_get_intersection_line:NNNNNNNN}
%  \begin{syntax}
%    \cs{trt_lt_get_intersection_line:NNNNNNNN}\meta{a1}\meta{b1}\meta{c1}\meta{a2}\meta{b2}\meta{c2}\meta{x}\meta{y}
%  \end{syntax}
%  This function finds the intersection of lines $a_1x+b_1y=c_1$ and 
%  $a_2x+b_2y=c_2$, afterwards store the dimensions of the intersection in 
%  variables \meta{x} and \meta{y}.
%
%  All arguments are floating points variables, \meta{x} and \meta{y} needs to 
%  be local variables.
%\end{function}
%\begin{function}{\trt_lt_get_intersection_coordinate:nnnnNN}
%  \begin{syntax}
%    \cs{trt_lt_get_intersection_coordinate:nnnnNN}
%    \marg{coor 1}\marg{coor 2}\marg{coor 3}\marg{coor 4}\meta{x}\meta{y}
%  \end{syntax}
%  This function is a wrapper of \cs{trt_lt_get_intersection_line:NNNNNNNN}. It 
%  finds the intersection of the line joining \meta{coor 1}, \meta{coor 2} and 
%  the line joining \meta{coor 3}, \meta{coor 4}. The dimensions of the returned 
%  point is stored in \meta{x} and \meta{y}, which are local \pkg{l3fp} 
%  variables.
%
%  A warning will be raised if the lines are parallel, in that case \meta{x} and 
%  \meta{y} are set to zero.
%
%  This is the base of |intersection|.
%\end{function}
%\begin{codeexample}[]
%\begin{tikzpicture}
%  \draw (0,0) coordinate (a) -- (0,1) coordinate (b);
%  \draw (1,0) coordinate (c) -- (.5,1) coordinate (d);
%  \ExplSyntaxOn
%  \fp_new:N \l_foo_tmpa_fp
%  \fp_new:N \l_foo_tmpb_fp
%  \trt_lt_get_intersection_coordinate:nnnnNN {a} {b} {c} {d}
%    \l_foo_tmpa_fp \l_foo_tmpb_fp
%  \coordinate (i) at (\fp_to_dim:N \l_foo_tmpa_fp, 
%                      \fp_to_dim:N \l_foo_tmpb_fp);
%  \ExplSyntaxOff
%  \fill[maincolor] (i) circle (1.5pt);
%\end{tikzpicture}
%\end{codeexample}
%\ExplSyntaxOn
%\cs_undefine:N \l_foo_tmpa_fp
%\cs_undefine:N \l_foo_tmpb_fp
%\ExplSyntaxOff
%\begin{function}{\trt_lt_get_perpendicular_equation:nNNNNNN}
%  \begin{syntax}
%    \cs{trt_lt_get_perpendicular_equation:nNNNNNN}\marg{coor 1}\meta{a1}\meta{b1}\meta{c1}\meta{a2}\meta{b2}\meta{c2}
%  \end{syntax}
%  This function finds the line of equation $a_2x+b_2y=c_2$ that passes 
%  coordinate \meta{coor 1} and is perpendicular to $a_1x+b_1y=c_1$.
%\end{function}
%\begin{codeexample}[]
%\begin{tikzpicture}
%  \draw (-1,0) coordinate (b) node[left] {$(-1,0)$} -- 
%        (3,1)  coordinate (c) node[right] {$(3,1)$};
%  \fill (0.5,4) circle (1.5pt) coordinate (a) node[above] {$(0.5,4)$};
%  \ExplSyntaxOn
%  \fp_new:N \l_foo_tmpa_fp
%  \fp_new:N \l_foo_tmpb_fp
%  \fp_new:N \l_foo_tmpc_fp
%  \fp_new:N \l_foo_tmpd_fp
%  \fp_new:N \l_foo_tmpe_fp
%  \fp_new:N \l_foo_tmpf_fp
%  \trt_lt_get_line_equation:nnNNN {b} {c} 
%    \l_foo_tmpa_fp \l_foo_tmpb_fp \l_foo_tmpc_fp
%  \trt_lt_get_perpendicular_equation:nNNNNNN {a}
%    \l_foo_tmpa_fp \l_foo_tmpb_fp \l_foo_tmpc_fp
%    \l_foo_tmpd_fp \l_foo_tmpe_fp \l_foo_tmpf_fp
%  \def \resultequation {
%    $\fp_eval:n {round (\l_foo_tmpd_fp / 1cm)}x + 
%     \fp_eval:n {round (\l_foo_tmpe_fp / 1cm)}y
%    =\fp_eval:n {round (\l_foo_tmpf_fp / (1cm * 1cm))}$
%  }
%  \ExplSyntaxOff
%  \path (1,0) node[below=3mm,align=center] 
%    {Equation of perpendicular line:\\\resultequation};
%  \ExplSyntaxOn
%  \path (1,-1.5) node[below,align=left] {
%    Actual ~ values:\\
%    $a = \fp_eval:n {\l_foo_tmpd_fp / 1cm}$\\
%    $b = \fp_eval:n {\l_foo_tmpe_fp / 1cm}$\\
%    $c = \fp_eval:n {\l_foo_tmpf_fp / (1cm * 1cm)}$
%  };
%  \ExplSyntaxOff
%\end{tikzpicture}
%\end{codeexample}
%\ExplSyntaxOn
%\cs_undefine:N \l_foo_tmpa_fp
%\cs_undefine:N \l_foo_tmpb_fp
%\cs_undefine:N \l_foo_tmpc_fp
%\cs_undefine:N \l_foo_tmpd_fp
%\cs_undefine:N \l_foo_tmpe_fp
%\cs_undefine:N \l_foo_tmpf_fp
%\ExplSyntaxOff
%\begin{function}{\trt_lt_get_perpendicular_coordinate:nnnNN}
%  \begin{syntax}
%    \cs{trt_lt_get_perpendicular_coordinate:nnnNN}\marg{coor 1}\marg{coor 2}\marg{coor 3}\meta{x}\meta{y}
%  \end{syntax}
%  Find the dimensions of the foot of perpendicular from \meta{coor 1} to the 
%  line joining \meta{coor 2} and \meta{coor 3}. Afterwards store the 
%  dimensions found in \meta{x} and \meta{y}.
%
%  This is the base of |foot of perpendicular|.
%\end{function}
%\subsubsection{The barycentric coordinate system}
%\begin{function}{initialize barycentric}
%  \begin{syntax}
%    /tikz/triangletools/initialize barycentric=\parg{coor 1}\parg{coor 2}\parg{coor 3}
%  \end{syntax}
%  Use the three coordinates as ``anchors'' of the barycentric coordinate 
%  system.
%\end{function}
%\begin{function}{bc3}
%  \begin{syntax}
%    (bc3 cs:\meta{l1},\meta{l2},\meta{l3})
%  \end{syntax}
%  Using the barycentric coordinate system. Note that the system needs to be 
%  initialized in advance using |initialize barycentric|, and an error message 
%  will be reported if you do otherwise.
%
%  The sum of \meta{l1}, \meta{l2} and \meta{l3} is not necessarily $1$ -- 
%  the package will take care of that internally.
%\end{function}
%\begin{codeexample}[]
%\begin{tikzpicture}
%  \draw (1,3) coordinate (a) --
%        (0,0) coordinate (b) --
%        (4,0) coordinate (c) -- cycle;
%  \fill[maincolor,trt={initialize barycentric=(a)(b)(c)}] (bc3 cs:1,2,3)
%    circle[radius=1.5pt] node[above] {$P_a$};
%  \fill[maincolor,trt={initialize barycentric=(b)(c)(a)}] (bc3 cs:1,2,3)
%    circle[radius=1.5pt] node[above] {$P_b$};
%  \fill[maincolor,trt={initialize barycentric=(c)(a)(b)}] (bc3 cs:1,2,3)
%    circle[radius=1.5pt] node[above] {$P_c$};
%\end{tikzpicture}
%\end{codeexample}
%\subsubsection{Distance-finding utility}
%\begin{function}{\trt_distance:nnN}
%  \begin{syntax}
%    \cs{trt_distance:nnN} \marg{coor 1} \marg{coor 2} \meta{fp var}
%  \end{syntax}
%  Find distance between \meta{coor 1} and \meta{coor 2}, and store that 
%  value to \meta{fp var}.
%\end{function}
%\begin{codeexample}[]
%\begin{tikzpicture}
%  \path (0,0) coordinate (a) (4,3) coordinate (b);
%  \ExplSyntaxOn
%  \fp_new:N \l_foo_tmpa_fp
%  \trt_distance:nnN {a} {b} \l_foo_tmpa_fp
%  \draw (a) -- (b) node[midway,sloped,below]
%    { \fp_eval:n {round(\l_foo_tmpa_fp / 1cm)} cm };
%  \ExplSyntaxOff
%\end{tikzpicture}
%\end{codeexample}
%\ExplSyntaxOn
%\cs_undefine:N \l_foo_tmpa_fp
%\ExplSyntaxOff
%\begin{function}{\trt_distance_triangle:nnnNNN}
%  \begin{syntax}
%    \cs{trt_distance_triangle:nnnNNN}\marg{coor 1}\marg{coor 2}\marg{coor 3}\meta{a}\meta{b}\meta{c}
%  \end{syntax}
%  \cs{trt_distance:nnN} is needed to find the side lengths in a triangle
%  (these side lengths are very helpful in many areas, for instance in this 
%  package to find special points based on the barycentric system). However, 
%  using it three times in a row is not quite elegant; this function is defined 
%  to automate that process.
%
%  \meta{a} is set to the distance between \meta{coor 2} and \meta{coor 3}, 
%  similar things happen for \meta{b} and \meta{c}.
%\end{function}
%\subsection{Customization}\label{customization}
%\begin{function}{output name}
%\begin{syntax}
%  /tikz/triangletools/output name=\meta{name}
%\end{syntax}
%This key can be used to change the name of the returned coordinates. The 
%initial value of this key is |trt output|.
%\end{function}
%\begin{codeexample}[]
%\begin{tikzpicture}[trt={output name=hello world}]
%  \draw (1,3) coordinate (a) -- 
%        (0,0) coordinate (b) -- 
%        (4,0) coordinate (c) -- cycle;
%  \fill[maincolor,trt={circumcenter=(a)(b)(c)}] (hello world) 
%    circle[radius=1.5pt] node[below] {$X$};
%\end{tikzpicture}
%\end{codeexample}
%
% \end{documentation}
%
% \StopEventually{\PrintIndex}
%
% \begin{implementation}
%
% \section{Implementation}
%
%    \begin{macrocode}
%<@@=trt>
%    \end{macrocode}
%
% \subsection{The main package file}
%
%    \begin{macrocode}
%<*triangletools>
\RequirePackage{tikz}
\RequirePackage{expl3}
\ProvidesExplPackage {triangletools} {2020/04/30} {0.1}
  {TikZ support for triangular geometry}
%    \end{macrocode}
%
% \begin{macro}{\trt@tmp@i, \trt@tmp@ii}
% We will use these dimensions many times to extract the dimensions of a \tikzname\ 
% coordinate.
%    \begin{macrocode}
\newdimen\trt@tmp@i
\newdimen\trt@tmp@ii
%    \end{macrocode}
% \end{macro}
%
% Let's load the necessary subpackage files.
%
%    \begin{macrocode}
\input {trtmessages.code.tex}
\input {trtlinetools.code.tex}
\input {trtbarycentric.code.tex}
\input {trtdistance.code.tex}
\input {trtspecialpoints.code.tex}
\input {trtfrontend.code.tex}
%</triangletools>
%    \end{macrocode}
%
% \subsection{Errors and warnings}
%
%    \begin{macrocode}
%<*messages>
\ProvidesExplFile {trtmessages.code.tex} {2020/04/30} {0.1}
  {The ~ triangletools ~ package: ~ Messages}
%    \end{macrocode}
%
% We also need to declare some helpful messages that we will use later on.
%
% In \file{trtlinetools.code.tex}, when we find the intersection of two lines, 
% a warning will be shown if the lines are parallel. The warning is based on 
% |intersection-not-found|.
%
%    \begin{macrocode}
\msg_new:nnnn {triangletools} {intersection-not-found}
  {
    Intersection ~ not ~ found.
  }
  {
    You ~ told ~ me ~ to ~ find ~ the ~ intersection ~ of ~ the ~ line ~ 
    joining ~ #1 ~ and ~ #2 ~ and ~ the ~ line ~ joining ~ #3 ~ and ~ #4, ~ 
    however ~ these ~ lines ~ are ~ parallel ~ so ~ I ~ can't ~ find ~ any ~ 
    intersection. ~ The ~ return ~ point ~ is ~ set ~ to ~ the ~ origin ~ 
    (0, ~ 0).
  }
%    \end{macrocode}
%
% When the barycentric coordinate system, implemented in 
% \file{trtbarycentric.code.tex}, is used, it should already be initialized, 
% \emph{i.e.} we should already know what are the three ``anchor'' coordinates 
% of the system. If the coordinate system is not yet initialized, this error 
% will be shown.
%
%    \begin{macrocode}
\msg_new:nnnn {triangletools} {uninitialized}
  {
    Barycentric ~ coordinate ~ system ~ not ~ initialized.
  }
  {
    You ~ have ~ not ~ initialized ~ the ~ three ~ anchor ~ points ~ for ~ 
    the ~ coordinate ~ system. ~ Please ~ initialize ~ the ~ points ~ 
    before ~ using ~ the ~ `bc3' ~ coordinate ~ system.
  }
%    \end{macrocode}
%
% We do let the user to find triangle center $X_i$ for any $i$. However this 
% package obviously can't implement all points in ETC (in fact, I will implement 
% only some most important points). An error will be raised if the user tries to 
% use an unimplemented point.
%
%    \begin{macrocode}
\msg_new:nnnn {triangletools} {center-not-found}
  {
    Triangle ~ center ~ not ~ found.
  }
  {
    I ~ can't ~ find ~ the ~ requested ~ triangle ~ center, ~ because ~ 
    point ~ X(#1) ~ is ~ not ~ yet ~ implemented ~ in ~ the ~ triangletools ~ 
    package. ~ Try ~ to ~ construct ~ it ~ yourself.
  }
%    \end{macrocode}
%
% We need to guard against using \cs{trtradius} before the macro stores 
% something.
%
%    \begin{macrocode}
\msg_new:nnnn {triangletools} {no-radius-found}
  {
    No ~ circles ~ can ~ be ~ constructed.
  }
  {
    I ~ can't ~ construct ~ the ~ requested ~ circle, ~ because ~ you ~ have ~ 
    not ~ request ~ me ~ to ~ construct ~ any ~ triangle ~ centers ~ that ~ 
    are ~ associated ~ to ~ a ~ circle. ~ I ~ will ~ set ~ 
    \protect\trtradius\space to ~ zero ~ now.
  }
%</messages>
%    \end{macrocode}
%
% \subsection{The backend layer}
% \subsubsection{The line tools utility}
%    \begin{macrocode}
%<*linetools>
\ProvidesExplFile {trtlinetools.code.tex} {2020/04/30} {0.1}
  {The ~ triangletools ~ package: ~ Utilities ~ for ~ lines}
%    \end{macrocode}
% In |trtlinetools.code.tex|, we will implement the necessary functions to 
% handle lines in a mathematical way.
%
% Firstly, let's declare some internal variables that we will use later.
%
% \begin{variable}{
%   \l_@@_lt_pointi_x_fp, \l_@@_lt_pointi_y_fp,
%   \l_@@_lt_pointii_x_fp, \l_@@_lt_pointii_y_fp
% }
% These variables are used to store coordinates of the two vertices of a 
% segment.
%    \begin{macrocode}
\fp_new:N \l_@@_lt_pointi_x_fp
\fp_new:N \l_@@_lt_pointi_y_fp
\fp_new:N \l_@@_lt_pointii_x_fp
\fp_new:N \l_@@_lt_pointii_y_fp
%    \end{macrocode}
% \end{variable}
%
% \begin{variable}{
%   \l_@@_lt_linei_a_fp, \l_@@_lt_linei_b_fp, \l_@@_lt_linei_c_fp, 
%   \l_@@_lt_lineii_a_fp, \l_@@_lt_lineii_b_fp, \l_@@_lt_lineii_c_fp
% }
% We will store the line equation under the format of $ax+by=c$, because this is 
% the most generic format. These six variables will do that job.
%
%    \begin{macrocode}
\fp_new:N \l_@@_lt_linei_a_fp
\fp_new:N \l_@@_lt_linei_b_fp
\fp_new:N \l_@@_lt_linei_c_fp
\fp_new:N \l_@@_lt_lineii_a_fp
\fp_new:N \l_@@_lt_lineii_b_fp
\fp_new:N \l_@@_lt_lineii_c_fp
%    \end{macrocode}
% \end{variable}
%
% \begin{variable}{\l_@@_lt_tmp_fp, \l_@@_lt_tmpa_fp, \l_@@_lt_tmpb_fp}
% Some additional temporary variables.
%    \begin{macrocode}
\fp_new:N \l_@@_lt_tmp_fp
\fp_new:N \l_@@_lt_tmpa_fp
\fp_new:N \l_@@_lt_tmpb_fp
%    \end{macrocode}
% \end{variable}
%
% \begin{macro}{\trt_lt_get_line_equation:nnNNN}
% Find the equation of the line passing |#1| and |#2|, and store the values of 
% $a,b,c$ found to |#3|, |#4| and |#5|, which are floating point variables, 
% respectively.
%    \begin{macrocode}
\cs_new:Npn \trt_lt_get_line_equation:nnNNN #1 #2 #3 #4 #5
  {
    \pgfextractx {\trt@tmp@i} {\pgfpointanchor{#1}{center}}
    \pgfextracty {\trt@tmp@ii} {\pgfpointanchor{#1}{center}}
    \fp_set:Nn \l_@@_i_pointi_x_fp {\trt@tmp@i}
    \fp_set:Nn \l_@@_i_pointi_y_fp {\trt@tmp@ii}
    \pgfextractx {\trt@tmp@i} {\pgfpointanchor{#2}{center}}
    \pgfextracty {\trt@tmp@ii} {\pgfpointanchor{#2}{center}}
    \fp_set:Nn \l_@@_i_pointii_x_fp {\trt@tmp@i}
    \fp_set:Nn \l_@@_i_pointii_y_fp {\trt@tmp@ii}
%    \end{macrocode}
% There is a simple hack here. We have $ax_1+by_1=c=ax_2+by_2$, which is 
% equivalent to $a(x_1-x_2)=b(y_2-y_1)$. Therefore $a=y_2-y_1$ and $b=x_1-x_2$ 
% can be used.
%    \begin{macrocode}
    \fp_set:Nn #3
      {
        \l_@@_i_pointii_y_fp - \l_@@_i_pointi_y_fp
      }
    \fp_set:Nn #4
      {
        \l_@@_i_pointi_x_fp - \l_@@_i_pointii_x_fp
      }
    \fp_set:Nn #5
      {
        #3 * \l_@@_i_pointi_x_fp + #4 * \l_@@_i_pointi_y_fp
      }
  }
%    \end{macrocode}
% \end{macro}
%
% \begin{macro}{\trt_lt_get_intersection_line:NNNNNNNN}
% Find the intersection of two lines with given equation, after that store the 
% intersection coordinate to floating point variables |#7| and |#8|.
%    \begin{macrocode}
\cs_new:Npn \trt_lt_get_intersection_line:NNNNNNNN #1 #2 #3 #4 #5 #6 #7 #8
  {
    \fp_set:Nn \l_@@_lt_tmp_fp { #1 * #5 - #4 * #2 }
%    \end{macrocode}
% If \cs{l_@@_lt_tmp_fp} is zero, the two lines are parallel. In that case, we 
% will issue a warning, and set the intersection coordinate to $(0,0)$. 
% Otherwise, continue computing as usual.
%    \begin{macrocode}
    \fp_compare:nNnTF {\l_@@_lt_tmp_fp} = {0}
      {
        \msg_warning:nnnnnn {triangletools} {intersection-not-found}
          {(#1)} {(#2)} {(#3)} {(#4)}
        \fp_set:Nn #7 {0}
        \fp_set:Nn #8 {0}
      }
      {
        \fp_set:Nn #7 { ( #5 * #3 - #2 * #6 ) / \l_@@_lt_tmp_fp }
        \fp_set:Nn #8 { ( #1 * #6 - #4 * #3 ) / \l_@@_lt_tmp_fp }
      }
  }
%    \end{macrocode}
% \end{macro}
%
% \begin{macro}{\trt_lt_get_intersection_coordinate:nnnnNN}
% Let's generalize \cs{trt_lt_get_intersection_line:NNNNNNNN}. The following 
% function finds the intersection of two lines between |#1|, |#2| and |#3|, 
% |#4|, and store the coordinates to |#5| and |#6|. We still use floating 
% point variables here, as they might be useful in the future.
%    \begin{macrocode}
\cs_new:Npn \trt_lt_get_intersection_coordinate:nnnnNN #1 #2 #3 #4 #5 #6
  {
    \trt_lt_get_line_equation:nnNNN {#1} {#2} 
      \l_@@_lt_linei_a_fp \l_@@_lt_linei_b_fp \l_@@_lt_linei_c_fp
    \trt_lt_get_line_equation:nnNNN {#3} {#4} 
      \l_@@_lt_lineii_a_fp \l_@@_lt_lineii_b_fp \l_@@_lt_lineii_c_fp
    \trt_lt_get_intersection_line:NNNNNNNN
      \l_@@_lt_linei_a_fp \l_@@_lt_linei_b_fp \l_@@_lt_linei_c_fp
      \l_@@_lt_lineii_a_fp \l_@@_lt_lineii_b_fp \l_@@_lt_lineii_c_fp
      #5 #6
  }
%    \end{macrocode}
% \end{macro}
%
% \begin{macro}{\@@_lt_return_intersection:nnnnn}
% Now, let's \tikzname ify the above function! Note that I use |overlay| because 
% I don't want to affect the bounding box. The user can use the returned 
% coordinate to change the bounding box in whatever way he wants to.
%    \begin{macrocode}
\cs_new:Npn \@@_lt_return_intersection:nnnnn #1 #2 #3 #4 #5
  {
    \trt_lt_get_intersection_coordinate:nnnnNN {#1} {#2} {#3} {#4}
      \l_@@_lt_tmpa_fp \l_@@_lt_tmpb_fp
    \coordinate[overlay] (#5) at 
      (\fp_to_dim:N \l_@@_lt_tmpa_fp, \fp_to_dim:N \l_@@_lt_tmpb_fp);
  }
%    \end{macrocode}
% \end{macro}
% Next, let's make some implementation regarding perpendicularity.
%
% \begin{macro}{\trt_lt_get_perpendicular_equation:nNNNNNN}
% This function finds the equation of the line passing point and being 
% perpendicular to a line having a given equation. The task is not quite 
% complicated: note that lines $ax+by=c$ and $ay-bx=d$ are perpendicular.
%    \begin{macrocode}
\cs_new:Npn \trt_lt_get_perpendicular_equation:nNNNNNN #1 #2 #3 #4 #5 #6 #7
  {
    \fp_set:Nn #5 { -#3 }
    \fp_set:Nn #6 { #2 }
    \pgfextractx {\trt@tmp@i} {\pgfpointanchor{#1}{center}}
    \pgfextracty {\trt@tmp@ii} {\pgfpointanchor{#1}{center}}
    \fp_set:Nn \l_@@_lt_tmpa_fp {\trt@tmp@i}
    \fp_set:Nn \l_@@_lt_tmpb_fp {\trt@tmp@ii}
    \fp_set:Nn #7 { #5 * \l_@@_lt_tmpa_fp + #6 * \l_@@_lt_tmpb_fp }
  }
%    \end{macrocode}
% \end{macro}
%
% \begin{macro}{\trt_lt_get_perpendicular_coordinate:nnnNN}
% The base implemented, let's find the foot of perpendicular from a point to 
% a segment.
%    \begin{macrocode}
\cs_new:Npn \trt_lt_get_perpendicular_coordinate:nnnNN #1 #2 #3 #4 #5
  {
    \trt_lt_get_line_equation:nnNNN {#2} {#3} 
      \l_@@_lt_linei_a_fp \l_@@_lt_linei_b_fp \l_@@_lt_linei_c_fp

    \trt_lt_get_perpendicular_equation:nNNNNNN {#1} 
      \l_@@_lt_linei_a_fp \l_@@_lt_linei_b_fp \l_@@_lt_linei_c_fp
      \l_@@_lt_lineii_a_fp \l_@@_lt_lineii_b_fp \l_@@_lt_lineii_c_fp

    \trt_lt_get_intersection_line:NNNNNNNN
      \l_@@_lt_linei_a_fp \l_@@_lt_linei_b_fp \l_@@_lt_linei_c_fp
      \l_@@_lt_lineii_a_fp \l_@@_lt_lineii_b_fp \l_@@_lt_lineii_c_fp
      #4 #5
  }
%    \end{macrocode}
% \end{macro}
%
% \begin{macro}{\@@_lt_return_perpendicular_coordinate:nnnn}
% This is just a wrapper of \cs{trt_lt_get_perpendicular_coordinate:nnnNN}.
%    \begin{macrocode}
\cs_new:Npn \@@_lt_return_perpendicular_coordinate:nnnn #1 #2 #3 #4
  {
    \trt_lt_get_perpendicular_coordinate:nnnNN {#1} {#2} {#3} 
      \l_@@_lt_tmpa_fp \l_@@_lt_tmpb_fp
    \coordinate[overlay] (#4) at 
      (\fp_to_dim:N \l_@@_lt_tmpa_fp, \fp_to_dim:N \l_@@_lt_tmpb_fp);
  }
%</linetools>
%    \end{macrocode}
% \end{macro}
%
% \subsubsection{The barycentric coordinate system utility}
%    \begin{macrocode}
%<*barycentric>
\ProvidesExplFile {trtbarycentric.code.tex} {2020/04/30} {0.1}
  {
    The ~ triangletools ~ package: ~ Utilities ~ for ~ the ~ barycentric ~ 
    coordinate ~ system.
  }
%    \end{macrocode}
% In \file{trtbarycentric.code.tex}, we will implement the three-point 
% barycentric coordinate system, which is essential in constructing many 
% special points in a triangle.^^A TODO: n points?
%
% \begin{variable}{
%   \l_@@_bc_anchor_ix_fp, \l_@@_bc_anchor_iy_fp,
%   \l_@@_bc_anchor_iix_fp, \l_@@_bc_anchor_iiy_fp,
%   \l_@@_bc_anchor_iiix_fp, \l_@@_bc_anchor_iiiy_fp
% }
% We use six variables to store the `anchors' of the barycentric 
% coordinate system.
%    \begin{macrocode}
\fp_new:N \l_@@_bc_anchor_ix_fp
\fp_new:N \l_@@_bc_anchor_iy_fp
\fp_new:N \l_@@_bc_anchor_iix_fp
\fp_new:N \l_@@_bc_anchor_iiy_fp
\fp_new:N \l_@@_bc_anchor_iiix_fp
\fp_new:N \l_@@_bc_anchor_iiiy_fp
%    \end{macrocode}
% \end{variable}
%
% \begin{variable}{
%   \l_@@_bc_lambda_i_fp, \l_@@_bc_lambda_ii_fp, \l_@@_bc_lambda_iii_fp
% }
% We use these variables to store the user input coordinate. Note that our 
% system is a three-point one, hence exactly three number is required.
%
% Why lambda $\lambda$? Well, I don't know. Wikipedia uses that, so I do the 
% same.
%    \begin{macrocode}
\fp_new:N \l_@@_bc_lambda_i_fp
\fp_new:N \l_@@_bc_lambda_ii_fp
\fp_new:N \l_@@_bc_lambda_iii_fp
%    \end{macrocode}
% \end{variable}
%
% \begin{variable}{\l_@@_bc_initialized_bool}
% We need to guard against using the system before initializing. This boolean 
% variable does that job: if it is set to |false| (default), do nothing.
%    \begin{macrocode}
\bool_new:N \l_@@_bc_initialized_bool
%    \end{macrocode}
% \end{variable}
%
% \begin{variable}{\l_@@_bc_tmp_fp}
% A temporary variable.
%    \begin{macrocode}
\fp_new:N \l_@@_bc_tmp_fp
%    \end{macrocode}
% \end{variable}
%
% \begin{macro}{\@@_bc_initialize:nnn}
% Initialize the barycentric coordinate system. This is the only place where 
% \cs{l_@@_bc_initialized_bool} can be set to true, so this function must be 
% executed before everything else in this file.
%    \begin{macrocode}
\cs_new:Npn \@@_bc_initialize:nnn #1 #2 #3
  {
    \bool_set_true:N \l_@@_bc_initialized_bool
    \pgfextractx {\trt@tmp@i} {\pgfpointanchor{#1}{center}}
    \pgfextracty {\trt@tmp@ii} {\pgfpointanchor{#1}{center}}
    \fp_set:Nn \l_@@_bc_anchor_ix_fp {\trt@tmp@i}
    \fp_set:Nn \l_@@_bc_anchor_iy_fp {\trt@tmp@ii}
    \pgfextractx {\trt@tmp@i} {\pgfpointanchor{#2}{center}}
    \pgfextracty {\trt@tmp@ii} {\pgfpointanchor{#2}{center}}
    \fp_set:Nn \l_@@_bc_anchor_iix_fp {\trt@tmp@i}
    \fp_set:Nn \l_@@_bc_anchor_iiy_fp {\trt@tmp@ii}
    \pgfextractx {\trt@tmp@i} {\pgfpointanchor{#3}{center}}
    \pgfextracty {\trt@tmp@ii} {\pgfpointanchor{#3}{center}}
    \fp_set:Nn \l_@@_bc_anchor_iiix_fp {\trt@tmp@i}
    \fp_set:Nn \l_@@_bc_anchor_iiiy_fp {\trt@tmp@ii}
  }
%    \end{macrocode}
% \end{macro}
% \begin{macro}{bc3}
% The |bc3| coordinate system implementation. We will guard against using it 
% when \cs{@@_bc_initialize:nnn} is not yet executed -- in that case, 
% |uninitialized| error will be raised.
%
% We will receive arguments of |bc3| as |#1,#2,#3|, so a simple parser is 
% needed. All interesting things will be done with that parser.
%
%    \begin{macrocode}
\tikzdeclarecoordinatesystem {bc3}
  {
    \bool_if:NTF \l_@@_bc_initialized_bool
      {
        \@@_bc_parse:w #1 \q_stop
      }
      {
        \msg_error:nn {triangletools} {uninitialized}
      }
  }
%    \end{macrocode}
% \end{macro}
% \begin{macro}{\@@_bc_parse:w}
% This is the parser we use for |bc3|.
%
% The conversion from $\lambda_i$ to the Cartesian format is pretty simple, 
% we have $x=\lambda_1x_1+\lambda_2x_2+\lambda_3x_3$ and the formula for $y$ is 
% similar. However, first we have to change the value of $\lambda_i$ so that 
% $\lambda_1+\lambda_2+\lambda_3=1$.
%    \begin{macrocode}
\cs_new:Npn \@@_bc_parse:w #1,#2,#3 \q_stop
  {
    \fp_set:Nn \l_@@_bc_tmp_fp { (#1) + (#2) + (#3) }
    \fp_set:Nn \l_@@_bc_lambda_i_fp { (#1) / (\l_@@_bc_tmp_fp) }
    \fp_set:Nn \l_@@_bc_lambda_ii_fp { (#2) / (\l_@@_bc_tmp_fp) }
    \fp_set:Nn \l_@@_bc_lambda_iii_fp { (#3) / (\l_@@_bc_tmp_fp) }
    \fp_set:Nn \l_@@_tmp_a_fp
      {
        \l_@@_bc_anchor_ix_fp * \l_@@_bc_lambda_i_fp + 
        \l_@@_bc_anchor_iix_fp * \l_@@_bc_lambda_ii_fp + 
        \l_@@_bc_anchor_iiix_fp * \l_@@_bc_lambda_iii_fp
      }
    \fp_set:Nn \l_@@_tmp_b_fp
      {
        \l_@@_bc_anchor_iy_fp * \l_@@_bc_lambda_i_fp + 
        \l_@@_bc_anchor_iiy_fp * \l_@@_bc_lambda_ii_fp + 
        \l_@@_bc_anchor_iiiy_fp * \l_@@_bc_lambda_iii_fp
      }
%    \end{macrocode}
% Floating point variables are not \TeX\ dimensions, hence \cs{fp_to_dim:N} is 
% used.
%    \begin{macrocode}
    \pgf@x = \fp_to_dim:N \l_@@_tmp_a_fp
    \pgf@y = \fp_to_dim:N \l_@@_tmp_b_fp
  }
%</barycentric>
%    \end{macrocode}
% \end{macro}
%
% \subsubsection{Distance-finding utility}
%
%    \begin{macrocode}
%<*distance>
\ProvidesExplFile {trtdistance.code.tex} {2020/04/30} {0.1}
  {The ~ triangletools ~ package: ~ Utilities ~ for ~ 2d ~ distance}
%    \end{macrocode}
%
% This file implements functions to find the distance between (2d) \tikzname\ 
% coordinates.
%
% \begin{variable}{
%   \l_@@_d_pointi_x_fp, \l_@@_d_pointi_y_fp, 
%   \l_@@_d_pointii_x_fp, \l_@@_d_pointii_y_fp
% }
% These variables are used to store the coordinates of the points between which 
% we are finding the distance.
%    \begin{macrocode}
\fp_new:N \l_@@_d_pointi_x_fp
\fp_new:N \l_@@_d_pointii_x_fp
\fp_new:N \l_@@_d_pointi_y_fp
\fp_new:N \l_@@_d_pointii_y_fp
%    \end{macrocode}
% \end{variable}
%
% \begin{macro}{\trt_distance:nnN}
% Find the distance between \tikzname\ coordinates |#1| and |#2|.
%    \begin{macrocode}
\cs_new:Npn \trt_distance:nnN #1 #2 #3
  {
    \pgfextractx {\trt@tmp@i} {\pgfpointanchor{#1}{center}}
    \pgfextracty {\trt@tmp@ii} {\pgfpointanchor{#1}{center}}
    \fp_set:Nn \l_@@_d_pointi_x_fp {\trt@tmp@i}
    \fp_set:Nn \l_@@_d_pointi_y_fp {\trt@tmp@ii}
    \pgfextractx {\trt@tmp@i} {\pgfpointanchor{#2}{center}}
    \pgfextracty {\trt@tmp@ii} {\pgfpointanchor{#2}{center}}
    \fp_set:Nn \l_@@_d_pointii_x_fp {\trt@tmp@i}
    \fp_set:Nn \l_@@_d_pointii_y_fp {\trt@tmp@ii}
    \fp_set:Nn #3
      {
        sqrt((
          (\l_@@_d_pointi_x_fp - \l_@@_d_pointii_x_fp) * 
          (\l_@@_d_pointi_x_fp - \l_@@_d_pointii_x_fp)
        ) + (
          (\l_@@_d_pointi_y_fp - \l_@@_d_pointii_y_fp) * 
          (\l_@@_d_pointi_y_fp - \l_@@_d_pointii_y_fp)
        ))
      }
  }
%    \end{macrocode}
% \end{macro}
%
% \begin{macro}{\trt_distance_triangle:nnnNNN}
% We mainly need the above function to find the side length in a triangle. Let's 
% create a function that do so automatically.
%    \begin{macrocode}
\cs_new:Npn \trt_distance_triangle:nnnNNN #1 #2 #3 #4 #5 #6
  {
    \trt_distance:nnN {#2} {#3} #4
    \trt_distance:nnN {#3} {#1} #5
    \trt_distance:nnN {#1} {#2} #6
  }
%</distance>
%    \end{macrocode}
% \end{macro}
%
% \subsection{Construction of triangle centers}
%
%    \begin{macrocode}
%<*specialpoints>
\ProvidesExplFile {trtspecialpoints.code.tex} {2020/04/30} {0.1}
  {The ~ triangletools ~ package: ~ Triangle ~ center ~ construction}
%    \end{macrocode}
%
% This file will use the utility implemented in the above sections to find 
% some most important triangle centers described in the ETC.
%
% \begin{variable}{\l_@@_sp_a_fp, \l_@@_sp_b_fp, \l_@@_sp_c_fp}
% We will need the side length of the triangle for some centers.
%    \begin{macrocode}
\fp_new:N \l_@@_sp_a_fp
\fp_new:N \l_@@_sp_b_fp
\fp_new:N \l_@@_sp_c_fp
%    \end{macrocode}
% \end{variable}
%
% \begin{variable}{
%   \l_@@_sp_coordinatei_x_fp, \l_@@_sp_coordinatei_y_fp,
%   \l_@@_sp_coordinateii_x_fp, \l_@@_sp_coordinateii_y_fp,
%   \l_@@_sp_coordinateiii_x_fp, \l_@@_sp_coordinateiii_y_fp,
%   \l_@@_sp_linei_a_fp, \l_@@_sp_linei_b_fp, \l_@@_sp_linei_c_fp,
%   \l_@@_sp_lineii_a_fp, \l_@@_sp_lineii_b_fp, \l_@@_sp_lineii_c_fp
% }
% These variables may also be helpful for triangle centers for which a simple 
% formula doesn't exist, e.g. the circumcenter.
%    \begin{macrocode}
\fp_new:N \l_@@_sp_coordinatei_x_fp
\fp_new:N \l_@@_sp_coordinatei_y_fp
\fp_new:N \l_@@_sp_coordinateii_x_fp
\fp_new:N \l_@@_sp_coordinateii_y_fp
\fp_new:N \l_@@_sp_coordinateiii_x_fp
\fp_new:N \l_@@_sp_coordinateiii_y_fp
\fp_new:N \l_@@_sp_linei_a_fp
\fp_new:N \l_@@_sp_linei_b_fp
\fp_new:N \l_@@_sp_linei_c_fp
\fp_new:N \l_@@_sp_lineii_a_fp
\fp_new:N \l_@@_sp_lineii_b_fp
\fp_new:N \l_@@_sp_lineii_c_fp
%    \end{macrocode}
% \end{variable}
%
% \begin{variable}{\l_@@_sp_tmpa_fp, \l_@@_sp_tmpb_fp, \l_@@_sp_tmpc_fp}
% Some additional temporary variables.
%    \begin{macrocode}
\fp_new:N \l_@@_sp_tmpa_fp
\fp_new:N \l_@@_sp_tmpb_fp
\fp_new:N \l_@@_sp_tmpc_fp
%    \end{macrocode}
% \end{variable}
%
% \subsubsection{$X_1$ -- The incenter}
% Each center will have a function taking four arguments. The first three 
% arguments are the \tikzname\ coordinates of the triangle vertices; the last argument 
% is the name of the return \tikzname\ coordinate.
%
% To prevent conflict between these sister functions when they are used together, 
% I put each of them inside a \TeX\ group.
%
% \begin{macro}{\trt_sp_incenter:nnnn}
% Return the incenter. It is based on the barycentric coordinate of the incenter, 
% $(a,b,c)$.
%    \begin{macrocode}
\cs_new:Npn \trt_sp_incenter:nnnn #1 #2 #3 #4
  {
    \group_begin:
      \@@_bc_initialize:nnn {#1} {#2} {#3}
      \trt_distance_triangle:nnnNNN {#1} {#2} {#3}
        \l_@@_sp_a_fp \l_@@_sp_b_fp \l_@@_sp_c_fp
      \path[overlay] (bc3 ~ cs \c_colon_str
        \fp_eval:n {\l_@@_sp_a_fp},
        \fp_eval:n {\l_@@_sp_b_fp},
        \fp_eval:n {\l_@@_sp_c_fp}) coordinate (#4);
    \group_end:
  }
%    \end{macrocode}
% \end{macro}
%
% \begin{macro}{\trt_sp_excenter:nnnn}
% Return the excenter of the triangle, with respect to vertex |#1|. This center 
% is just a derivation of the incenter; also it is not unique, so it is not 
% assigned a number. Barycentric coordinate of the excenter is $(-a,b,c)$, where 
% $a$ is the length of the side joining |#2| and |#3|.
%
% Note that this is the only function in this series in which argument order is 
% important.
%
%    \begin{macrocode}
\cs_new:Npn \trt_sp_excenter:nnnn #1 #2 #3 #4
  {
    \group_begin:
      \@@_bc_initialize:nnn {#1} {#2} {#3}
      \trt_distance_triangle:nnnNNN {#1} {#2} {#3}
        \l_@@_sp_a_fp \l_@@_sp_b_fp \l_@@_sp_c_fp
      \path[overlay] (bc3 ~ cs \c_colon_str
        \fp_eval:n {- \l_@@_sp_a_fp},
        \fp_eval:n {\l_@@_sp_b_fp},
        \fp_eval:n {\l_@@_sp_c_fp}) coordinate (#4);
    \group_end:
  }
%    \end{macrocode}
% \end{macro}
%
% \subsubsection{$X_2$ -- The centroid}
% \begin{macro}{\trt_sp_centroid:nnnn}
% This is perhaps the simplest of all. Barycentric coordinate: $(1,1,1)$.
%    \begin{macrocode}
\cs_new:Npn \trt_sp_centroid:nnnn #1 #2 #3 #4
  {
    \group_begin:
      \@@_bc_initialize:nnn {#1} {#2} {#3}
      \path[overlay] (bc3 ~ cs \c_colon_str 1, 1, 1) coordinate (#4);
    \group_end:
  }
%    \end{macrocode}
% \end{macro}
% \subsubsection{$X_3$ -- The circumcenter}
% \begin{macro}{\trt_sp_circumcenter:nnnn}
% This is opposite to $X_2$: perhaps this is the most complex of all. The 
% barycentric coordinate formula is not simple enough for me, so I construct 
% this point purely manually: find the intersection of the perpendicular 
% bisectors.
%    \begin{macrocode}
\cs_new:Npn \trt_sp_circumcenter:nnnn #1 #2 #3 #4
  {
    \group_begin:
%    \end{macrocode}
% Firstly, let's store the coordinate of the vertices.
%    \begin{macrocode}
      \pgfextractx {\trt@tmp@i} {\pgfpointanchor{#1}{center}}
      \pgfextracty {\trt@tmp@ii} {\pgfpointanchor{#1}{center}}
      \fp_set:Nn \l_@@_sp_coordinatei_x_fp {\trt@tmp@i}
      \fp_set:Nn \l_@@_sp_coordinatei_y_fp {\trt@tmp@ii}
      \pgfextractx {\trt@tmp@i} {\pgfpointanchor{#2}{center}}
      \pgfextracty {\trt@tmp@ii} {\pgfpointanchor{#2}{center}}
      \fp_set:Nn \l_@@_sp_coordinateii_x_fp {\trt@tmp@i}
      \fp_set:Nn \l_@@_sp_coordinateii_y_fp {\trt@tmp@ii}
      \pgfextractx {\trt@tmp@i} {\pgfpointanchor{#3}{center}}
      \pgfextracty {\trt@tmp@ii} {\pgfpointanchor{#3}{center}}
      \fp_set:Nn \l_@@_sp_coordinateiii_x_fp {\trt@tmp@i}
      \fp_set:Nn \l_@@_sp_coordinateiii_y_fp {\trt@tmp@ii}
%    \end{macrocode}
% Now, let's change point |#2| to the midpoint between |#1| and |#2|, and do the 
% same for |#3|.
%    \begin{macrocode}
      \fp_set:Nn \l_@@_sp_coordinateii_x_fp
        {
          (\l_@@_sp_coordinatei_x_fp + \l_@@_sp_coordinateii_x_fp) / 2
        }
      \fp_set:Nn \l_@@_sp_coordinateii_y_fp
        {
          (\l_@@_sp_coordinatei_y_fp + \l_@@_sp_coordinateii_y_fp) / 2
        }
      \coordinate[overlay] (trt@tmp@ii) at (
        \fp_to_dim:N \l_@@_sp_coordinateii_x_fp, 
        \fp_to_dim:N \l_@@_sp_coordinateii_y_fp);
      \fp_set:Nn \l_@@_sp_coordinateiii_x_fp
        {
          (\l_@@_sp_coordinatei_x_fp + \l_@@_sp_coordinateiii_x_fp) / 2
        }
      \fp_set:Nn \l_@@_sp_coordinateiii_y_fp
        {
          (\l_@@_sp_coordinatei_y_fp + \l_@@_sp_coordinateiii_y_fp) / 2
        }
      \coordinate[overlay] (trt@tmp@iii) at (
        \fp_to_dim:N \l_@@_sp_coordinateiii_x_fp, 
        \fp_to_dim:N \l_@@_sp_coordinateiii_y_fp);
%    \end{macrocode}
% All we have to do now is to find the equations of the bisectors and their 
% intersection.
%    \begin{macrocode}
      \trt_lt_get_line_equation:nnNNN {#1} {#2}
        \l_@@_sp_tmpa_fp \l_@@_sp_tmpb_fp \l_@@_sp_tmpc_fp
      \trt_lt_get_perpendicular_equation:nNNNNNN {trt@tmp@ii}
        \l_@@_sp_tmpa_fp \l_@@_sp_tmpb_fp \l_@@_sp_tmpc_fp
        \l_@@_sp_linei_a_fp \l_@@_sp_linei_b_fp \l_@@_sp_linei_c_fp
      \trt_lt_get_line_equation:nnNNN {#1} {#3}
        \l_@@_sp_tmpa_fp \l_@@_sp_tmpb_fp \l_@@_sp_tmpc_fp
      \trt_lt_get_perpendicular_equation:nNNNNNN {trt@tmp@iii}
        \l_@@_sp_tmpa_fp \l_@@_sp_tmpb_fp \l_@@_sp_tmpc_fp
        \l_@@_sp_lineii_a_fp \l_@@_sp_lineii_b_fp \l_@@_sp_lineii_c_fp
      \trt_lt_get_intersection_line:NNNNNNNN
        \l_@@_sp_linei_a_fp \l_@@_sp_linei_b_fp \l_@@_sp_linei_c_fp
        \l_@@_sp_lineii_a_fp \l_@@_sp_lineii_b_fp \l_@@_sp_lineii_c_fp
        \l_@@_sp_tmpa_fp \l_@@_sp_tmpb_fp
      \coordinate[overlay] (#4) at (
        \fp_to_dim:N \l_@@_sp_tmpa_fp, \fp_to_dim:N \l_@@_sp_tmpb_fp);
    \group_end:
  }
%    \end{macrocode}
% Quite surprisingly, the function is still very fast after all this. On my 
% machine it never exceeds 10ms in execution time.
% \end{macro}
% \subsubsection{$X_4$ -- The orthocenter}
% \begin{macro}{\trt_sp_orthocenter:nnnn}
% Return the orthocenter of the triangle. This point is also constructed 
% manually instead of using a proved formula. However, the utilities help 
% making the construction look very simple.
%    \begin{macrocode}
\cs_new:Npn \trt_sp_orthocenter:nnnn #1 #2 #3 #4
  {
    \group_begin:
      \@@_lt_return_perpendicular_coordinate:nnnn {#1} {#2} {#3} {trt@tmp@i}
      \@@_lt_return_perpendicular_coordinate:nnnn {#2} {#1} {#3} {trt@tmp@ii}
      \@@_lt_return_intersection:nnnnn
        {#1} {trt@tmp@i} {#2} {trt@tmp@ii} {#4}
    \group_end:
  }
%    \end{macrocode}
% \end{macro}
% \subsubsection{$X_5$ -- The nine-point center}
% \begin{macro}{\trt_sp_ninepointcenter:nnnn}
% Return the center of the nine-point circle.
%    \begin{macrocode}
\cs_new:Npn \trt_sp_ninepointcenter:nnnn #1 #2 #3 #4
  {
    \group_begin:
%    \end{macrocode}
% $X_5$ is is the midpoint of $X_3$ and $X_4$. Therefore, for simplicity, 
% $X_3$ and $X_4$ are constructed first. This causes some run-time overhead, 
% however the overall execution time is still below 15ms, which is, in my 
% opinion, still good.
%
% Note that we already used |trt@tmp@i| and |trt@tmp@ii| coordinates in the 
% construction of $X_3$ and $X_4$, so to prevent conflict, |trt@tmp@iii| and 
% |trt@tmp@iv| are used.
%    \begin{macrocode}
      \trt_sp_circumcenter:nnnn {#1} {#2} {#3} {trt@tmp@iii}
      \trt_sp_orthocenter:nnnn {#1} {#2} {#3} {trt@tmp@iv}
      \pgfextractx {\trt@tmp@i} {\pgfpointanchor{trt@tmp@iii}{center}}
      \pgfextracty {\trt@tmp@ii} {\pgfpointanchor{trt@tmp@iii}{center}}
      \fp_set:Nn \l_@@_sp_tmpa_fp {\trt@tmp@i}
      \fp_set:Nn \l_@@_sp_tmpb_fp {\trt@tmp@ii}
      \pgfextractx {\trt@tmp@i} {\pgfpointanchor{trt@tmp@iv}{center}}
      \pgfextracty {\trt@tmp@ii} {\pgfpointanchor{trt@tmp@iv}{center}}
      \fp_set:Nn \l_@@_sp_tmpa_fp { (\trt@tmp@i + \l_@@_sp_tmpa_fp) / 2 }
      \fp_set:Nn \l_@@_sp_tmpb_fp { (\trt@tmp@ii + \l_@@_sp_tmpb_fp) / 2 }
      \coordinate[overlay] (#4) at (\fp_to_dim:N \l_@@_sp_tmpa_fp,
        \fp_to_dim:N \l_@@_sp_tmpb_fp);
    \group_end:
  }
%    \end{macrocode}
% \end{macro}
% \subsubsection{$X_6$ -- The symmedian point}
% \begin{macro}{\trt_sp_symmedian:nnnn}
% Return the symmedian point (\emph{aka.} the Lemoine point or Grebe point). 
% The barycentric coordinate of the point is $(a^2,b^2,c^2)$.
%    \begin{macrocode}
\cs_new:Npn \trt_sp_symmedian:nnnn #1 #2 #3 #4
  {
    \group_begin:
      \@@_bc_initialize:nnn {#1} {#2} {#3}
      \trt_distance_triangle:nnnNNN {#1} {#2} {#3}
        \l_@@_sp_a_fp \l_@@_sp_b_fp \l_@@_sp_c_fp
      \path[overlay] (bc3 ~ cs \c_colon_str
        \fp_eval:n {\l_@@_sp_a_fp * \l_@@_sp_a_fp},
        \fp_eval:n {\l_@@_sp_b_fp * \l_@@_sp_b_fp},
        \fp_eval:n {\l_@@_sp_c_fp * \l_@@_sp_c_fp}) coordinate (#4);
    \group_end:
  }
%    \end{macrocode}
% \end{macro}
% \subsubsection{$X_7$ -- The Gergonne point}
% \begin{macro}{\trt_sp_gergonne:nnnn}
% Return the Gergonne point of the triangle. The barycentric coordinate of the 
% point is $(\frac{1}{b+c-a},\frac{1}{c+a-b},\frac{1}{a+b-c})$.
%    \begin{macrocode}
\cs_new:Npn \trt_sp_gergonne:nnnn #1 #2 #3 #4
  {
    \group_begin:
      \@@_bc_initialize:nnn {#1} {#2} {#3}
      \trt_distance_triangle:nnnNNN {#1} {#2} {#3}
        \l_@@_sp_a_fp \l_@@_sp_b_fp \l_@@_sp_c_fp
      \path[overlay] (bc3 ~ cs \c_colon_str
        \fp_eval:n { 1/(\l_@@_sp_b_fp + \l_@@_sp_c_fp - \l_@@_sp_a_fp) },
        \fp_eval:n { 1/(\l_@@_sp_c_fp + \l_@@_sp_a_fp - \l_@@_sp_b_fp) },
        \fp_eval:n { 1/(\l_@@_sp_a_fp + \l_@@_sp_b_fp - \l_@@_sp_c_fp) }
      ) coordinate (#4);
    \group_end:
  }
%    \end{macrocode}
% \end{macro}
% \subsubsection{$X_8$ -- The Nagel point}
% \begin{macro}{\trt_sp_nagel:nnnn}
% Return the Nagel point. The barycentric coordinate of the point is 
% $(b+c-a,c+a-b,a+b-c)$.
%    \begin{macrocode}
\cs_new:Npn \trt_sp_nagel:nnnn #1 #2 #3 #4
  {
    \group_begin:
      \@@_bc_initialize:nnn {#1} {#2} {#3}
      \trt_distance_triangle:nnnNNN {#1} {#2} {#3}
        \l_@@_sp_a_fp \l_@@_sp_b_fp \l_@@_sp_c_fp
      \path[overlay] (bc3 ~ cs \c_colon_str
        \fp_eval:n { \l_@@_sp_b_fp + \l_@@_sp_c_fp - \l_@@_sp_a_fp },
        \fp_eval:n { \l_@@_sp_c_fp + \l_@@_sp_a_fp - \l_@@_sp_b_fp },
        \fp_eval:n { \l_@@_sp_a_fp + \l_@@_sp_b_fp - \l_@@_sp_c_fp }
      ) coordinate (#4);
    \group_end:
  }
%    \end{macrocode}
% \end{macro}
% \subsubsection{$X_9$ -- The \emph{mittenpunkt}}
% \begin{macro}{\trt_sp_mittenpunkt:nnnn}
% Return the \emph{mittenpunkt} of the triangle -- its barycentric coordinate is 
% $(a\times (b+c-a),b\times (c+a-b),c\times (a+b-c))$.
%    \begin{macrocode}
\cs_new:Npn \trt_sp_mittenpunkt:nnnn #1 #2 #3 #4
  {
    \group_begin:
      \@@_bc_initialize:nnn {#1} {#2} {#3}
      \trt_distance_triangle:nnnNNN {#1} {#2} {#3}
        \l_@@_sp_a_fp \l_@@_sp_b_fp \l_@@_sp_c_fp
      \path[overlay] (bc3 ~ cs \c_colon_str
        \fp_eval:n 
          { 
            \l_@@_sp_a_fp * (
              \l_@@_sp_b_fp + \l_@@_sp_c_fp - \l_@@_sp_a_fp
            )
          },
        \fp_eval:n 
          { 
            \l_@@_sp_b_fp * (
              \l_@@_sp_c_fp + \l_@@_sp_a_fp - \l_@@_sp_b_fp
            )
          },
        \fp_eval:n 
          { 
            \l_@@_sp_c_fp * (
              \l_@@_sp_a_fp + \l_@@_sp_b_fp - \l_@@_sp_c_fp
            )
          }
      ) coordinate (#4);
    \group_end:
  }
%    \end{macrocode}
% \end{macro}
% \subsubsection{$X_{10}$ -- The Spieker point}
% \begin{macro}{\trt_sp_spieker:nnnn}
% Return the Spieker point. The barycentric coordinate of the point is 
% $(b+c,c+a,a+b)$.
%    \begin{macrocode}
\cs_new:Npn \trt_sp_spieker:nnnn #1 #2 #3 #4
  {
    \group_begin:
      \@@_bc_initialize:nnn {#1} {#2} {#3}
      \trt_distance_triangle:nnnNNN {#1} {#2} {#3}
        \l_@@_sp_a_fp \l_@@_sp_b_fp \l_@@_sp_c_fp
      \path[overlay] (bc3 ~ cs \c_colon_str
        \fp_eval:n { \l_@@_sp_b_fp + \l_@@_sp_c_fp },
        \fp_eval:n { \l_@@_sp_c_fp + \l_@@_sp_a_fp },
        \fp_eval:n { \l_@@_sp_a_fp + \l_@@_sp_b_fp }
      ) coordinate (#4);
    \group_end:
  }
%</specialpoints>
%    \end{macrocode}
% \end{macro}
%
% \subsection{The frontend layer}
%    \begin{macrocode}
%<*frontend>
\ProvidesExplFile {trtfrontend.code.tex} {2020/04/30} {0.1}
  {The ~ triangletools ~ package: ~ The ~ front-end ~ layer}
%    \end{macrocode}
%
% The user interface of the package, which consists solely of \pkg{pgf} keys, 
% will be implemented in this file.
%
% \begin{variable}{\l_@@_fr_output_name_tl}
% Store the name of the output coordinate. Default to \verb*|trt output|.
%    \begin{macrocode}
\tl_new:N \l_@@_fr_output_name_tl
\tl_set:Nn \l_@@_fr_output_name_tl {trt ~ output}
%    \end{macrocode}
% \end{variable}
% \begin{variable}{\l_@@_fr_center_number_int}
% We only provide specific key for $X_1$, $X_2$, $X_3$ and $X_4$. All other 
% points can be referenced using a single generic key. We need to store the 
% index of that point so that we can choose the right function for the point.
%    \begin{macrocode}
\int_new:N \l_@@_fr_center_number_int
%    \end{macrocode}
% \end{variable}
% \begin{macro}{\trtradius}
% \begin{variable}{\l_@@_fr_radius_fp}
% This macro will store the radius if a circle is associated. Of course firstly 
% we need a floating point variable specified for that purpose.
%    \begin{macrocode}
\fp_new:N \l_@@_fr_radius_fp
\cs_gset_nopar:Npn \trtradius
  {
    \msg_error:nn {triangletools} {no-radius-found} 0pt
  }
%    \end{macrocode}
% \end{variable}
% \end{macro}
% \begin{macro}{trt}
% Now it's time for the keys. They will be stored under |/tikz/triangletools| 
% and can be accessed at |trt=|\marg{keys}.
%    \begin{macrocode}
\tikzset {
  triangletools/.is ~ family,
  trt/.code={\pgfkeys{/tikz/triangletools/.cd,#1}},
  triangletools/.cd,
%    \end{macrocode}
% \end{macro}
% \begin{macro}{output name}
% Change the output name of all returned coordinates.
%    \begin{macrocode}
  output ~ name/.code={
    \tl_set:Nn \l_@@_fr_output_name_tl {#1}
  },
%    \end{macrocode}
% \end{macro}
% \begin{macro}{intersection, foot of perpendicular}
% The front-end of the line tools utility.
%    \begin{macrocode}
  intersection/.code ~ args={(#1)(#2)--(#3)(#4)}{
    \@@_lt_return_intersection:nnnnn {#1} {#2} {#3} {#4} 
      {\tl_use:N \l_@@_fr_output_name_tl}
  },
  foot ~ of ~ perpendicular/.code ~ args={(#1)--(#2)(#3)}{
    \@@_lt_return_perpendicular_coordinate:nnnn {#1} {#2} {#3}
      {\tl_use:N \l_@@_fr_output_name_tl}
  },
%    \end{macrocode}
% \end{macro}
% \begin{macro}{initialize barycentric}
% The front-end of the barycentric coordinate system.
%    \begin{macrocode}
  initialize ~ barycentric/.code ~ args={(#1)(#2)(#3)}{
    \@@_bc_initialize:nnn {#1} {#2} {#3}
  },
%    \end{macrocode}
% \end{macro}
% \begin{macro}{incenter, excenter, centroid, circumcenter, orthocenter}
% The front-end of the triangle centers $X_1$ to $X_4$ and the excenter.
%    \begin{macrocode}
  incenter/.code ~ args={(#1)(#2)(#3)}{
    \trt_sp_incenter:nnnn {#1} {#2} {#3} {trt@tmp@center}
    \pgfkeysalso{foot ~ of ~ perpendicular=(trt@tmp@center)--(#1)(#2)}
    \trt_distance:nnN {\tl_use:N \l_@@_fr_output_name_tl} {trt@tmp@center} 
      \l_@@_fr_radius_fp
    \cs_gset_nopar:Npx \trtradius { \fp_to_dim:N \l_@@_fr_radius_fp }
    \coordinate (\tl_use:N \l_@@_fr_output_name_tl) at (trt@tmp@center);
  },
  excenter/.code ~ args={(#1)(#2)(#3)}{
    \trt_sp_excenter:nnnn {#1} {#2} {#3} {trt@tmp@center}
    \pgfkeysalso{foot ~ of ~ perpendicular=(trt@tmp@center)--(#2)(#3)}
    \trt_distance:nnN {\tl_use:N \l_@@_fr_output_name_tl} {trt@tmp@center} 
      \l_@@_fr_radius_fp
    \cs_gset_nopar:Npx \trtradius { \fp_to_dim:N \l_@@_fr_radius_fp }
    \coordinate (\tl_use:N \l_@@_fr_output_name_tl) at (trt@tmp@center);
  },
  centroid/.code ~ args={(#1)(#2)(#3)}{
    \trt_sp_centroid:nnnn {#1} {#2} {#3}
      {\tl_use:N \l_@@_fr_output_name_tl}
  },
  circumcenter/.code ~ args={(#1)(#2)(#3)}{
    \trt_sp_circumcenter:nnnn {#1} {#2} {#3}
      {\tl_use:N \l_@@_fr_output_name_tl}
    \trt_distance:nnN {\tl_use:N \l_@@_fr_output_name_tl} {#1} 
      \l_@@_fr_radius_fp
    \cs_gset_nopar:Npx \trtradius { \fp_to_dim:N \l_@@_fr_radius_fp }
  },
  orthocenter/.code ~ args={(#1)(#2)(#3)}{
    \trt_sp_orthocenter:nnnn {#1} {#2} {#3}
      {\tl_use:N \l_@@_fr_output_name_tl}
  },
%    \end{macrocode}
% \end{macro}
% \begin{macro}{triangle center}
% This key is used to access all centers. I don't give any centers from $X_5$ 
% a key -- this key is necessary to construct them.
%    \begin{macrocode}
  triangle ~ center/.code ~ args={(#1)(#2)(#3)(#4)}{
    \int_case:nnF {#4}
      {
        {1} {
          \pgfkeysalso{incenter=(#1)(#2)(#3)}
        }
        {2} {
          \pgfkeysalso{centroid=(#1)(#2)(#3)}
        }
        {3} {
          \pgfkeysalso{circumcenter=(#1)(#2)(#3)}
        }
        {4} {
          \pgfkeysalso{orthocenter=(#1)(#2)(#3)}
        }
        {5} {
          \trt_sp_ninepointcenter:nnnn {#1} {#2} {#3}
            {\tl_use:N \l_@@_fr_output_name_tl}
          \group_begin:
            \@@_bc_initialize:nnn {#1} {#2} {#3}
            \coordinate (trt@tmp@mid) at (bc3 ~ cs \c_colon_str 1,1,0);
          \group_end:
          \trt_distance:nnN {\tl_use:N \l_@@_fr_output_name_tl} {trt@tmp@mid} 
            \l_@@_fr_radius_fp
          \cs_gset_nopar:Npx \trtradius { \fp_to_dim:N \l_@@_fr_radius_fp }
        }
        {6} {
          \trt_sp_symmedian:nnnn {#1} {#2} {#3}
            {\tl_use:N \l_@@_fr_output_name_tl}
        }
        {7} {
          \trt_sp_gergonne:nnnn {#1} {#2} {#3}
            {\tl_use:N \l_@@_fr_output_name_tl}
        }
        {8} {
          \trt_sp_nagel:nnnn {#1} {#2} {#3}
            {\tl_use:N \l_@@_fr_output_name_tl}
        }
        {9} {
          \trt_sp_mittenpunkt:nnnn {#1} {#2} {#3}
            {\tl_use:N \l_@@_fr_output_name_tl}
        }
        {10} {
          \trt_sp_spieker:nnnn {#1} {#2} {#3} {trt@tmp@center}
          \group_begin:
            \@@_bc_initialize:nnn {#1} {#2} {#3}
            \coordinate (trt@tmp@midi)  at (bc3 ~ cs \c_colon_str 1,1,0);
            \coordinate (trt@tmp@midii) at (bc3 ~ cs \c_colon_str 0,1,1);
          \group_end:
          \pgfkeysalso{
            foot~of~perpendicular=(trt@tmp@center)--(trt@tmp@midi)(trt@tmp@midii)
          }
          \trt_distance:nnN {\tl_use:N \l_@@_fr_output_name_tl} {trt@tmp@center}
            \l_@@_fr_radius_fp
          \cs_gset_nopar:Npx \trtradius { \fp_to_dim:N \l_@@_fr_radius_fp }
          \coordinate (\tl_use:N \l_@@_fr_output_name_tl) at (trt@tmp@center);
        }
      }
      {
        \msg_error:nnn {triangletools} {center-not-found} {#4}
      }
  }
}
%</frontend>
%    \end{macrocode}
% \end{macro}
%
% \end{implementation}
%
% \Finale